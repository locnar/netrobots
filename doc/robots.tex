\documentclass{article}
\title{Guidelines for the Netrobots C programming project}
\author{Paolo Bonzini, Adina Mosincat,
        \and
        adapted for FRC Team 4918 by Jeff Gibbons}
\date{\today}
\def\UL{\char`\_}

\begin{document}

\maketitle

\section{Game parameters}

\subsection{Battlefield}

        The battlefield is a 1,000 by 1,000 meter square.  A wall
        surrounds the perimeter, so that a robot running into the wall
        will incur damage.

        The lower left corner has the coordinates x = 0, y = 0; the upper
        right corner has the coordinated x = 999, y = 999.

        The compass system is oriented so that east (right) is 0
        degrees, 90 is north, 180 is west, 270 is south.  One degree
        below due east is 359.


\begin{verbatim}
     135    90   45
         \  |  / 
          \ | /
    180 --- x --- 0
          / | \
         /  |  \ 
     225   270  315
\end{verbatim}



\subsection{Robot offense}

        The main offensive weapons are the cannon and scanner.  The
        cannon has a range of 700 meters.  There are an unlimited number
        of missiles that can be fired, but a reloading factor limits the
        number of each robot's missiles in the air at any one time to two.
        The cannon is mounted on an independent turret, and therefore can
        fire any direction, 0-359, regardless of robot heading.

        The scanner is an optical device that can instantly scan any
        chosen heading, 0-359.  The scanner has a maximum beam-width of  
        +/- 10 degrees.  This enables the robot to quickly scan the field
        at a low resolution, then use maximum resolution to pinpoint an
        opponent.


\subsection{Robot defense }

        The only defenses available are the robot engine and status
        functions.  The motor can be engaged on any heading, 0-359, in
        speeds from 0-100 percent of power.  There are acceleration and
        deceleration factors.  A speed of 0 stops the motor.  Turns can
        be negotiated at speeds of 50% and less, in any direction.  Of
        course, the engine can be engaged any time, and is necessary
        on offense when a target is beyond the 700 meter range of the
        cannon.

        Certain status functions provide feedback to the robot. The
        primary functions indicate the percent of damage, and current x
        and y locations on the battlefield.  Another function provides
        current movement speed.


\subsection{Disabling opponents}

        A robot is considered dead when the damage reaches 100\%.  The
        amount of damage inflicted during the game is as follows:

\begin{description}
\item[2\%] upon collision into another robot (both robots in a    
                     collision receive damage) or into a wall.  A
                     collision also causes the robot engine to disengage,
                     and speed is reduced to 0.

\item[3\%] when a missile explodes within a 20--40 meter radius.

\item[5\%] when a missile explodes within a 5--20 meter radius.

\item[10\%] when a missile explodes within a 0--5 meter radius.
\end{description}

        Damage is cumulative, and cannot be repaired.  However, a robot
        does not lose any mobility, fire potential, etc. at high damage
        levels.  In other words, a robot at 99\% damage performs equally
        as a robot with no damage.

        A robot program that crashes or exits (!) will immediately bring
        the robot to 100\% damage.

\section{Simulation execution}

        The server is started and given on the command line the number of
        robots participating in the simulation.  When the given number of
        clients is connected, the simulation starts.

        The simulation proceeds in cycles.  The sequence of operations in
        a cycle is as follows:
        \begin{enumerate}
        \item Send current location to the robots
        \item Move and explode missiles
        \item Update display
        \item Wait 10 ms (minus the time that was used to update the display)
        \item Receive commands from the robots
        \item Send command replies to the robots
        \item Move robots
        \end{enumerate}

        Note that the above protocol is actually not exactly
        how the server and client interact.  The actual protocol is equivalent
	though.

        After a number of cycles (which may be given on the command line
        to the server) has passed, if there are 2 or more robots still functioning,
        the server declares a draw and causes the remaining clients to exit.

\section{Robots Function Library}

        The function library provides control of the robot.  The library
        is included from \texttt{robots.h}.  The first time a function from
        the library is called, the library will attempt to connect to a server
        (which must be already running).

\begin{description}
\item[\texttt{int scan(int degree,int resolution)}]
        invokes the robot's scanner, at a specified
        angle and resolution.  It returns 0 if no robots are
        within the scan range or a positive integer representing the
        range to the closest robot.  Angles not between 0 and 359 are accepted.
        Resolution controls the scanner's sensing resolution, up to
        $\pm 10$ degrees.

        Examples:
\begin{verbatim}
range = scan( 45, 0 );   /* scan 45, with no variance */
range = scan( 365, 10 ); /* scans the range from 355 to 15 */
\end{verbatim}

\item[\texttt{int cannon(int degree,int range)}]
        fires a missile heading a specified range
        and direction.  It returns 1 (true) if a missile was fired,
        or 0 (false) if the cannon is reloading.  ``Out of range'' angles
        are accepted as in \texttt{scan()}.  Range can be 0-700, with greater
        ranges truncated to 700.

        Example:
\begin{verbatim}
degree = 45;                          /* direction to look... */
if ( 0 < ( range = scan(degree,2) ) ) /* see if a target is there */
  cannon( degree, range );            /* if so, fire a missile */
\end{verbatim}

\item[\texttt{int drive(int degree,int speed)}]
        activates the engine, on a 
        specified heading and target speed.  Degree is forced into the range
        0-359 as in \texttt{scan()}.  Speed is expressed as a percent,
        with 100 as maximum.  A speed of 0 disengages the drive.  Changes in
        direction can be made only at speeds of less than 50 percent,
        otherwise the degree argument will be silently ignored (and \emph{only}
        the degree argument; the speed argument will be obeyed on the \emph{old}
        heading).

        Decelerating from speed 20 to speed 0 should take exactly 8 meters.

        Example:
\begin{verbatim}
drive( 0, 100 );  /* head due east, at maximum speed */
drive( 90, 0 );   /* stop motion */
\end{verbatim}

\item[\texttt{int damage()}]
        The damage() function returns the current amount of damage
        incurred.  \texttt{damage()} takes no arguments, and returns the percent
        of damage, 0-99. (100 percent damage means the robot is
        completely disabled, thus no longer running!) 

        Example:
\begin{verbatim}
d = damage();         /* save current state */
...                   /* other instructions */
if ( damage() != d )  /* compare current state to prior state */
{
  drive( 90, 100 );   /* robot has been hit, start moving */
  d = damage();       /* get current damage again */
} 
\end{verbatim}

\item[\texttt{void cycle()}]
        waits until the server has updated the robots' positions.
        This function is not needed except to avoid busy waiting.

\item[\texttt{int speed()}]
        returns the current speed of the robot.
        \texttt{speed()} takes no arguments, and returns the percent of speed,
        0-100.  Note that \texttt{speed()} may not always be the same as the
        last \texttt{drive()}, because of acceleration and deceleration.

        Example:
\begin{verbatim}
drive( 270, 100 );    /* start drive, due south */
...                   /* other instructions */
if ( speed() == 0 )   /* check current speed */
  drive( 90, 20 );    /* ran into south wall or another robot; */
                      /* get moving again!                     */
\end{verbatim}

\item[\texttt{int loc\UL x()} and \texttt{int loc\UL y()}]
        return the robot's current x (respectively, y) axis location.
        They take no arguments, and return 0-999.

        Example:
\begin{verbatim}
drive( 180, 50 );      /* start heading for west wall */
while ( 20 < loc_x() )
  cycle();             /* do nothing until we are close */
drive( 180, 0 );       /* stop drive, let robot coast to a stop */
\end{verbatim}
\end{description}

\end{document}
